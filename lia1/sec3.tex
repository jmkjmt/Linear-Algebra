\section{Elementary Matrix Operations and Systems of Linear Equations}
\subsection{Elementary Matrix Operations and Elementary Matrices}
\begin{theorem}
    Let \( A \in M_{m \times n}(\mathbb{F}) \), and suppose that \( B \) is obtained from \( A \) by performing an elementary row [column] operation. Then there exists an \( m \times m \) [\( n \times n \)] elementary matrix \( E \) such that \( B = EA \) [\( B = AE \)]. In fact, \( E \) is obtained from \( I_m \) [\( I_n \)] by performing the same elementary row [column] operation as that which was performed on \( A \) to obtain \( B \). Conversely, if \( E \) is an elementary \( m \times m \) [\( n \times n \)] matrix, then \( EA \) [\( AE \)] is the matrix obtained from \( A \) by performing the same elementary row [column] operation as that which produces \( E \) from \( I_m \) [\( I_n \)].
\end{theorem}

\begin{theorem}
    Elementary matrices are invertible, and the inverse of an elementary matrix is an elementary matrix of the same type.
\end{theorem}

\textbf{exercise}
\begin{enumerate}
    \item[9.] Prove that any elementary row [column] operation of type 1 can be obtained by a succession of three elementary row [column] operations of type 3 followed by one elementary row [column] operation of type 2. \vspace{5cm}
    \item[12.] Let \(A\) be an \(m \times n\) matrix. Prove that there exists a sequence of elementary row operations of types 1 and 3 that transforms \(A\) into an upper triangular matrix.
\end{enumerate}
\vspace{5cm}
\subsection{The Rank of a Matrix and Matrix Inverses}
\begin{theorem}
    Let \( T: V \to W \) be a linear transformation between finite-dimensional vector spaces, and let \( \beta \) and \( \gamma \) be ordered bases for \( V \) and \( W \), respectively. Then 
\[
\text{rank}(T) = \text{rank}([T]_\beta^\gamma).
\]
\end{theorem}
\vspace{5cm}
\begin{theorem}
    Let \( A \) be an \( m \times n \) matrix. If \( P \) and \( Q \) are invertible \( m \times m \) and \( n \times n \) matrices, respectively, then
\begin{enumerate}
    \item[(a)] \(\text{rank}(AQ) = \text{rank}(A),\)
    \item[(b)] \(\text{rank}(PA) = \text{rank}(A),\) \\
    and therefore,
    \item[(c)] \(\text{rank}(PAQ) = \text{rank}(A).\)
\end{enumerate}
\end{theorem}
\newpage
\begin{corollary}
    Elementary row and column operations on a matrix are rank preserving.
\end{corollary}
\vspace{3cm}
\begin{theorem}
    The rank of any matrix equals the maximum number of its linearly independent columns; that is, the rank of a matrix is the dimension of the subspace generated by its columns.
\end{theorem}
\vspace{5cm}
\begin{theorem}
    Let \( A \) be an \( m \times n \) matrix of rank \( r \). Then \( r \leq m \), \( r \leq n \), and by means of a finite number of elementary row and column operations, \( A \) can be transformed into the matrix
    \[
    D = \begin{pmatrix}
        I_r & O_1 \\
        O_2 & O_3
    \end{pmatrix},
    \]
    where \( O_1 \), \( O_2 \), and \( O_3 \) are zero matrices. Thus \( D_{ii} = 1 \) for \( i \leq r \) and \( D_{ij} = 0 \) otherwise.
\end{theorem}
\newpage

\begin{corollary}
    Let \( A \) be an \( m \times n \) matrix of rank \( r \). Then there exist invertible matrices \( B \) and \( C \) of sizes \( m \times m \) and \( n \times n \), respectively, such that \( D = BAC \) ,
    where 
    \[
    D  =  \begin{pmatrix}
        I_r & O_1 \\
        O_2 & O_3
    \end{pmatrix}
    \]
    is the \( m \times n \) matrix in which \( O_1 \), \( O_2 \), and \( O_3 \) are zero matrices.
\end{corollary}
\vspace{3cm}

\begin{corollary}
    Let \( A \) be an \( m \times n \) matrix. Then 
    \begin{enumerate}
        \item[(a)] \(\text{rank}(A^T) = \text{rank}(A),\)
        \item[(b)] The rank of any matrix equals the maximum number of its linearly independent rows; that is, the rank of a matrix is the dimension of the subspace generated by its rows.
        \item[(c)] The rows and columns of any matrix generate subspaces of the same dimension, numerically equal to the rank of the matrix.
    \end{enumerate}
\end{corollary}
\vspace{5cm}
\begin{corollary}
    Every invertible matrix is a product of elementary matrices.
\end{corollary}
\newpage
\begin{theorem}
    Let \( T \) : \( V \to W \) and \( U \) : \( W \to Z \) be linear transformations on finite-dimensional vector spaces \( V \), \( W \), and \( Z \), and let \( A \) and \( B \) be matrices such that the product \( AB \) is defined. Then
    \begin{enumerate}
        \item[(a)] \(\text{rank}(UT) \leq \text{rank}(U).\)
        \item[(b)] \(\text{rank}(UT) \leq \text{rank}(T).\)
        \item[(c)] \(\text{rank}(AB) \leq \text{rank}(A).\)
        \item[(d)] \(\text{rank}(AB) \leq \text{rank}(B).\)
    \end{enumerate}
\end{theorem}
\vspace{7cm}
\textbf{exercise}
\begin{enumerate}
        \item[14.] Let \(T, U: V \to W\) be linear transformations.
        \begin{enumerate}
            \item[(a)] Prove that \(R(T+U) \subseteq R(T) + R(U)\).  
            \item[(b)] Prove that if \(W\) is finite-dimensional, then \(\text{rank}(T+U) \leq \text{rank}(T) + \text{rank}(U)\).
            \item[(c)] Deduce from (b) that \(\text{rank}(A+B) \leq \text{rank}(A) + \text{rank}(B)\) for any \(m \times n\) matrices \(A\) and \(B\). 
        \end{enumerate}
        \vspace{7cm}
        \item[21.] Let \(A\) be an \(m \times n\) matrix with rank \(m\). Prove that there exists an \(n \times m \) matrix \(B\) such that \(AB = I_m\).
\end{enumerate} 
\vspace{3cm}

\subsection{Systems of Linear Equations---Theoretical Aspects}
\begin{theorem}
    Let \( Ax = 0 \) be a homogeneous system of \( m \) linear equations in \( n \) unknowns over a field \( F \). Let \( K \) denote the set of all solutions to \( Ax = 0 \). Then \( K =  N\)(\(L_A\));
    hence, \( K \) is a subspace of \( F^n \) of dimension \( n - \text{rank}(L_A) = n - \text{rank}(A) \).
\end{theorem}
\vspace{3cm}
\begin{corollary}
    If \( m < n \) , the system \( Ax = 0 \) has a nonzero solution.
\end{corollary}
\vspace{3cm}
\begin{theorem}
    Let \( K \) be the solution set of a system of linear equations \( Ax = b \), and let \( K_H \) be the soution set of the corresponding homogeneous system \( Ax = 0 \).
    Then for any solution \( s \) to \( Ax = b \) \\
    \[ K = \{s\} + K_H = \{s+k: k \in K_H\} \].
\end{theorem}
\newpage
\begin{theorem}
    Let \( Ax = b \) be a system of \( n \) linear equations in \( n \) unknowns. If \( A \) is invertible, then the system has exactly one solution, namely, \( A^{-1}b \).
    Conversely, if the system has exactly one solution, then \( A \) is invertible.
\end{theorem}
\vspace{7cm}
\begin{theorem}
    Let \( Ax = b \) be a system of linear equations. Then the system is consistent if and only if \(\text{rank}(A) = \text{rank}(A|b)\).
\end{theorem}
\vspace{5cm}
\begin{theorem}
    Let \( A \) be an \( n \times n \) input-output matrix having the form
    \[
    A = \begin{pmatrix}
        B & C \\
        D & E
        \end{pmatrix},
    \]
    where \( D \) is a \( 1 \times (n-1)\) positive vedctor and \( C \) is an \((n-1)\)\(\times 1\) positive vector. Then \((I - A)\)\(x = 0\) has a one-dimensional solution set that is generated by a nonnegative vector.
\end{theorem}
\newpage

\textbf{exercise}
\begin{enumerate}
    \item[10.] Prove or give a counterexample to the following statement: If the coefficient matrix of a system of \(m\) linear equations in \(n\) unknowns has rank \(m\), then the system has a solution.
\end{enumerate}
\vspace{3cm}
\subsection{Systems of Linear Equations---Computational Aspects}
\begin{theorem}
    Let \( Ax = b \) be a system of \( m \) linear equations in \( n \) unknowns, and let \( C \) be an invertible \( m \times n \) matrix. Then the system \((CA)\)\(x = Cb \) is equivalent to \( Ax = b \).
\end{theorem}
\vspace{4cm}
\begin{corollary}
    Let \( Ax = b \) be a system of \( m \) linear equations in \( n \) unknowns.
    If \((A'|b')\) is obtained from \((A |b)\) by a finite number of elementary row operations, then the system \( A'x = b' \) is equivalent to the original system.
\end{corollary}
\vspace{4cm}
\begin{theorem}
    Gaussian elimination transforms any matrix into its reduced row echelon form.
\end{theorem}
\begin{theorem}
    Let \(Ax = b \) be a system of \( r \) nonzero equations in \( n \) unknowns. Suppose that \(\text{rank}(A) = \text{rank}(A|b)\) and that \((A|b)\) is in reduced row echelon form. Then
    \begin{enumerate}
        \item[(a)] \(\text{rank}(A) = r \).
        \item[(b)] If the general solution obtained by the procedure above is of the form
        \[
        s = s_0 + t_1 u_1 + t_2 u_2 + \dots + t_{n-r} u_{n-r},
        \]
        then {\(u_1,u_2,\dots , u_{n-r}\)} is a basis for the solution set of the corresponding homogeneous system, and \( s_0 \) is a solution to the original system.
    \end{enumerate}
\end{theorem}
\vspace{5cm}
\begin{theorem}
    Let \(A\) be an \(m\times n\) matrix of rank \(r\), where \(r > 0\), and let \( B \) be the reduced row echelon form of \( A \). Then
\begin{enumerate}
    \item[(a)] The number of nonzero rows in \( B \) is \(r\).
    \item[(b)] For each \( i = 1, 2, \dots , r ,\) there is a column \(b_{j_i}\) of \(B\) such taht \(b_{j_i} = e_i \).
    \item[(c)] The columns of \( A \) numbered \( j_1, j_2, \dots , j_r \) are linearly independent.
    \item[(d)] For each \( k = 1, 2, \dots, n \), if column \(k\) of \(B \) is \(d_1 e_1 + d_2 e_2 + \dots + d_r e_r \),then column \( k \) of \(A \) is \(d_1 a_{j_1}+d_2 a_{j_2} + \dots + d_r a_{j_r} \).
\end{enumerate}
\end{theorem}
\vspace{7cm}
\begin{corollary}
    The reduce row echelon form of a matrix is unique.
\end{corollary}
\vspace{3cm}