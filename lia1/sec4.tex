\section{Determinants}
\subsection{Determinants of Order 2}
\begin{theorem}
    The function \(\det: M_{2 \times 2}(\mathbb{F}) \to \mathbb{F}\) is a linear function of each row of a \(2 \times 2\) matrix when the other row is held fixed. That is, if \(u, v, w\) are in \(\mathbb{F}^2\) and \(k\) is a scalar, then
    \[
    \det \begin{pmatrix} u + kv \\ w \end{pmatrix}
    = \det \begin{pmatrix} u \\ w \end{pmatrix} + k \det \begin{pmatrix} v \\ w \end{pmatrix}
    \]
    and
    \[
    \det \begin{pmatrix} w \\ u + kv \end{pmatrix}
    = \det \begin{pmatrix} w \\ u \end{pmatrix} + k \det \begin{pmatrix} w \\ v \end{pmatrix}.
    \]
\end{theorem}
\vspace{2cm}
\begin{theorem}
    Let \(A \in M_{2 \times 2}(\mathbb{F})\). Then the determinant of \(A\) is nonzero if and only if \(A\) is invertible. Moreover, if \(A\) is invertible, then
    \[
    A^{-1} = \frac{1}{\det(A)} 
    \begin{pmatrix}
        A_{22} & -A_{12} \\
        -A_{21} & A_{11}
    \end{pmatrix}.
    \]
\end{theorem}
\newpage
\subsection{Determinants of Order n}
\begin{theorem}
    The determinant of an \( n \times n \) matrix is a linear function of each row when the remaining rows are held fixed. That is, for \( 1 \leq r \leq n \), we have
    \[
    \det
    \begin{pmatrix}
        a_1 \\
        \vdots \\
        a_{r-1} \\
        u + kv \\
        a_{r+1} \\
        \vdots \\
        a_n
    \end{pmatrix}
    =
    \det
    \begin{pmatrix}
        a_1 \\
        \vdots \\
        a_{r-1} \\
        u \\
        a_{r+1} \\
        \vdots \\
        a_n
    \end{pmatrix}
    + k
    \det
    \begin{pmatrix}
        a_1 \\
        \vdots \\
        a_{r-1} \\
        v \\
        a_{r+1} \\
        \vdots \\
        a_n
    \end{pmatrix}
    \]
    whenever \(k \) is a scalar and \(u, v,\) and each \(a_i\) are row verctor in \(\mathbb{F}^n\).
\end{theorem}
\vspace{12cm}
\begin{corollary}
    If \(A \in M_{n \times n}(\mathbb{F})\) has a row consisting entirely of zeros, then \(\det (A) = 0\).
\end{corollary}
\newpage
\begin{lemma}
    Let \(B \in M_{n \times n}(\mathbb{F})\), where \( n \geq 2 \). If row \(i\) of \(B\) equals \(e_k\) for some \(k (1\leq k \leq n)\), then \(\det (B) = (-1)^{i+k} \det (\tilde{B}_{ik})\).
\end{lemma}
\newpage
\begin{theorem}
    The determinant of a square matrix can be evaluated by cofactor expansion along any row. That is, if \( A \in M_{n \times n}(\mathbb{F}) \), then for any integer \( i(1 \leq i \leq n ) \),
    \[
    \det (A) = \sum_{j=1}^{n} (-1)^{i+j} A_{ij} \det (\tilde{A}_{ij}).
    \]
\end{theorem}
\vspace{3cm}
\begin{corollary}
    If \( A \in M_{n \times n}(\mathbb{F}) \) has two identical rows, then \(\det (A) = 0\).
\end{corollary}
\vspace{5cm}
\begin{theorem}
    If \( A \in M_{n \times n}(\mathbb{F}) \) is matrix obtained form \(A\) by interchanging any two rows of \(A\), then \(\det (B) = - \det (A)\).
\end{theorem}
\newpage
\begin{theorem}
    Let \( A \in M_{n \times n}(\mathbb{F}) \), and let \( B \) be a matrix obtained by adding a multiple of one row of \(A\) to another row of \(A\). Then \(\det (B) = \det (A)\).
\end{theorem}
\vspace{3cm}
\begin{corollary}
    If \( A \in M_{n \times n}(\mathbb{F}) \) has rank less than \(n\), then \(\det (A) = 0\).
\end{corollary}
\vspace{3cm}
\textbf{exercise}
\begin{enumerate}
    \item[23.] Prove that the determinant of an upper triangula matrix is the product of its diagonal entries. \vspace{3cm}
    \item[26.] Let \(A \in M_{n \times n}(\mathbb{F})\). Under what conditions is \(\det(-A) = \det(A)\)? \vspace{3cm}
    \item[29.] Prove that if \(E\) is an elementary matrix, then \(\det(E^t) = \det(E)\). \vspace{3cm}
    \item[30.] Let the rows of \(A \in M_{n \times n}(\mathbb{F})\) be \(a_1, a_2, \dots , a_n\), and let \(B\) be the matrix in which the rows are \(a_n, a_{n-1}, \dots , a_1\). Calculate \(\det(B)\) in terms of \(\det(A)\).
\end{enumerate}
\vspace{3cm}
\subsection{Properties of Determinants}
\begin{theorem}
    For any \(A,B \in M_{n \times n}(F)\), \( \det (AB) = \det (A) \det (B) \).
\end{theorem}
\newpage
\begin{corollary}
    A matrix \(A \in M_{n \times n}(\mathbb{F})\) is invertible if and only if \(\det (A) \neq 0\).
    Furthermore, if \(A\) is invertible, then \(\det (A^{-1}) = \frac{1}{\det (A)}\).
\end{corollary}
\vspace{3cm}
\begin{theorem}
    For any \(A \in M_{n \times n}(\mathbb{F})\), \(\det (A^t) = \det (A)\).
\end{theorem}
\vspace{5cm}
\begin{theorem}[\textbf{Cramer's Rule}]
    Let \( Ax = b\) be the matrix form of a system of \( n\) linear equations in \( n \) unknowns, where \( x = (x_1, x_2, \dots , x_n)^t \).
    If \(\det (A) \neq 0\), then the system has a unique solution, and for each \(k\ (k = 1, 2, \dots , n)\),
    \[
    x_k = \frac{\det (M_k)}{\det (A)},
    \]
    where \(M_k\) is the \(n \times n \) matrix obtained from \(A\) by replacing the column \(k\) of \(A\) by \(b\).
\end{theorem}
\vspace{5cm}
\textbf{exercise}
\begin{enumerate}
    \item[12.] A matrix \(Q \in M_{n \times n}(\mathbb{R})\) is orthogonal if \(QQ^t = I\). Prove that if \(Q\) is orthogonal, then \(\det (Q) = \pm 1\). \vspace{3cm}
    \item[16.] Use determinants to prove that if \(A,B \in M_{n \times n}(\mathbb{F})\) are such that \(AB = I\), then \(A\) is invertible (and hence \(B = A^{-1}\)). \vspace{3cm}
    \item[20.] Suppose that \(M \in M_{n \times n}(\mathbb{F})\) can be written in the form \[ M = \begin{pmatrix}
        A & B \\
        0 & I
    \end{pmatrix},\] where \(A\) is a square matrix. Prove that det\((M) = \det(A)\). \vspace{5cm}
    \item[21.] Prove that if \(M \in M_{n \times n}(\mathbb{F})\) can be writte in the form \[ M = \begin{pmatrix}
        A & B \\
        0 & C
        \end{pmatrix},\] where \(A\) and \(C\) are square matrices, then \(\det (M) = \det (A) \det (C)\). \vspace{5cm}
    \item[23.] Let \(A \in M_{n \times n}(\mathbb{F})\) be nonzero. For any \(m (1 \leq m \leq n)\), and \(m \times m \) submatrix is obtained by deleting any \(n-m\)rows and and \(n-m\)columns of \(A\).
    \begin{enumerate}
        \item[(a)] Let \(k (1 \leq k \leq n)\) denote the largest integer such that some \(k \times k\) submatrix has a nonzero determinant. Prove that \(\text{rank}(A) = k\).
        \item[(b)] Conversely, suppose that \(\text{rank}(A) = k\). Prove that there exists a \(k \times k\) submatrix of \(A\) with a nonzero determinant.
    \end{enumerate}
    \vspace{5cm}
    \item[24.] Let \(A \in M_{n \times n}(\mathbb{F})\) have the form \[ A = \begin{pmatrix}
        0 &0& 0& \dots & 0& a_0 \\
        -1& 0& 0 &\dots & 0 &a_1 \\
        0 &-1& 0& \dots & 0& a_2 \\
        \vdots& \vdots& \vdots & & \vdots & \vdots \\
        0 &0 &0 &\dots& -1 & a_{n-1}
        \end{pmatrix}.\] Compute \(det(A+tI)\), where \(I\) is the \(n \times n\) identity matrix. \vspace{5cm}
\end{enumerate}


\subsection{Summary---Important Facts about Determinants}
\begin{quotation}{\textbf{Properties of the Determinant}}
    \begin{enumerate}
        \item If \(B \) is a matrix obtained by interchanging any two rows or interchanging any two columns of an \( n \times n \) matrix \( A \), then \( \det(B) = - \det(A) \).
        \item If \(B\) is a matrix obtained by multiplying each entroy of some row or coluomn of an \(n \times n \) matrix \(A\) by a scalr \(k\), then \(\det(B) = k \det(A)\).
        \item If \(B\) is a matrix obtained froam an \(n \times n \) matrix \(A\) by adding a multiple of row \(i\) to row \(j\) or a multiple of column \(i\) to column \(j\) for \(i \neq j\), then \(\det(B) = \det(A)\).
        \item The determinant of an upper triangular matrix is the product of its diagonal entries, In particular, \(\det(I) = 1\).
        \item If two rows (or columns) of a matrix are idential, then the determinant of the matrix is zero.
        \item For any \(n \times n \) matrices \(A \) and \(B\), \(\det(AB) = \det(A) \det(B)\).
        \item An \(n \times n\) matrix \(A\) is invertible if and only if \(\det(A) \neq 0 \). Furthermore, if \(A\) is invertible, then \(\det(A^{-1}) = \frac{1}{\det(A)}\).
        \item For any \(n \times n \) matrix \(A\), the determinant of \(A\) and \(A^t\) are equal.
        \item If \(A\) and \(B\) are similar matrices, then \(\det(A) = \det(B)\).
    \end{enumerate}
\end{quotation}
\newpage
\subsection{A Characterization of the Determinant}
\begin{definition}
    A function \(\delta: M_{n \times n}(\mathbb{F}) \to \mathbb{F}\) is called an \textbf{n-linear function} if it is a linear function of each row of an \( n \times n \) matrix when the remaining \( n-1\) rows are held fixed, that is , \(\delta\) is n-linear if, for every \(r=1,2,\dots ,n\), we have
    \[
    \delta \begin{pmatrix} a_1 \\ \vdots \\ a_{r-1} \\ u + kv \\ a_{r+1} \\ \vdots \\ a_n \end{pmatrix}
    = \delta \begin{pmatrix} a_1 \\ \vdots \\ a_{r-1} \\ u \\ a_{r+1} \\ \vdots \\ a_n \end{pmatrix}
    + k \delta \begin{pmatrix} a_1 \\ \vdots \\ a_{r-1} \\ v \\ a_{r+1} \\ \vdots \\ a_n \end{pmatrix}
    \]
    whenever \(k\) is a scalar and \(u, v,\) and each \(a_i\) are row vectors in \(\mathbb{F}^n\).
\end{definition}
\begin{definition}
    An n-linear function \(\delta: M_{n \times n}(\mathbb{F}) \to \mathbb{F}\) is called an \textbf{alternating} if, for each \(A \in M_{n \times n}(\mathbb{F})\), we have \(\delta(A) = 0\) whenever two adjacent rows of \(A\) are identical.
\end{definition}
\begin{theorem}
    Let \(\delta: M_{n \times n}(\mathbb{F}) \to \mathbb{F}\) be an alternating n-linear function.
    \begin{enumerate}
        \item[(a)] If \(A \in M_{n \times n}(\mathbb{F})\) and \(B\) is a matrix obtained from \(A\) by interchanging any two rows of \(A\), then \(\delta(B) = - \delta(A)\).
        \item[(b)] If \(A \in M_{n \times n}(\mathbb{F})\) has two identical rows, then \(\delta(A) = 0 \). 
    \end{enumerate}
\end{theorem}
\newpage
\begin{corollary}
    Let \(\delta: M_{n \times n}(\mathbb{F}) \to \mathbb{F}\) be an alternating n-linear function. If \(B\) is a matrix obtained from \(A \in M_{n \times n}(\mathbb{F})\) by adding a multiple of some row of \(A\) to another row, then \(\delta(B) = \delta(A)\).
\end{corollary}
\vspace{3cm}
\begin{corollary}
    Let \(\delta: M_{n \times n}(\mathbb{F}) \to \mathbb{F}\) be an alternating n-linear function. If \(M \in M_{n \times n}(\mathbb{F})\) has rank less than \(n\), then \(\delta(M) = 0\).
\end{corollary}
\vspace{2cm}
\begin{corollary}
    Let \(\delta: M_{n \times n}(\mathbb{F}) \to \mathbb{F}\) be an alternating n-linear function, and let \(E_1,E_2, and\ E_3\) in \(M_{n \times n}(\mathbb{F})\) be elementary matrices of types 1, 2, and 3, respectively. Suppoe that \(E_2\) is obtained by multiplying some row of \(I\) by the nonzero scalar \(k\). Then \(\delta(E_1) = -\delta(I)\), \(\delta(E_2) = k \delta(I)\), and \(\delta(E_3) = \delta(I)\).
\end{corollary}
\vspace{2cm}
\begin{theorem}
    Let \(\delta: M_{n \times n}(\mathbb{F}) \to \mathbb{F}\) be an alternating n-linear function such that \(\delta(I) = 1\). For any \(A,B \in M_{n \times n}(\mathbb{F})\), we have \(\delta(AB) = \delta(A) \delta(B)\).
\end{theorem}
\vspace{2cm}
\begin{theorem}
    If \(\delta: M_{n \times n}(\mathbb{F}) \to \mathbb{F}\) is an alternating n-linear function such that \(\delta(I) = 1\), then \(\delta(A) = \det(A)\) for every \(A \in M_{n \times n}(\mathbb{F})\).
\end{theorem}
\vspace{5cm}